\documentclass[10pt,a4paper]{article}
\usepackage[left=2cm,right=2cm,top=2cm,bottom=2cm]{geometry}
\usepackage[utf8]{inputenc}
\usepackage[french]{babel}
\usepackage[T1]{fontenc}
\usepackage[table]{xcolor}
\usepackage{listings}
\usepackage{algpseudocode}
\usepackage{fancyvrb}
\usepackage{fancybox}            %For \ovalbox
\usepackage{amsmath}
\usepackage{array}
\usepackage{amsfonts}
\usepackage{amsthm}
\usepackage{amssymb}
\usepackage{amsmath}
\usepackage{graphicx}
\usepackage{wasysym}
\usepackage{moreverb}
\usepackage{comment}
\usepackage{cancel}

\lstset{
    basicstyle=\ttfamily,
    escapeinside={||},
    xleftmargin=.1\textwidth,
    mathescape=true,
      literate=           %To have access to French accents in code blocks in lstlisting. For pseudocode and comments. See https://tex.stackexchange.com/a/85947
        {à}{{\`a}}1
        {â}{{\^a}}1
        {ä}{{\"a}}1
        {é}{{\'e}}1
        {è}{{\`e}}1
        {ê}{{\^e}}1
        {ë}{{\"e}}1
        {î}{{\^i}}1
        {ï}{{\"i}}1
        {ô}{{\^o}}1
        {ö}{{\"o}}1
        {ù}{{\`u}}1
        {û}{{\^u}}1
        {ü}{{\"u}}1
        {ÿ}{{\"y}}1
        {ç}{{\c{c}}}1
        {À}{{\`A}}1
        {Â}{{\^A}}1
        {Ä}{{\"A}}1
        {É}{{\'E}}1
        {È}{{\`E}}1
        {Ê}{{\^E}}1
        {Ë}{{\"E}}1
        {Î}{{\^I}}1
        {Ï}{{\"I}}1
        {Ô}{{\^O}}1
        {Ö}{{\"O}}1
        {Ù}{{\`U}}1
        {Û}{{\^U}}1
        {Ü}{{\"U}}1
        {Ÿ}{{\"Y}}1
        {Ç}{{\c{C}}}1
}

\usepackage{fancyhdr}
\pagestyle{fancy}
\fancyhead[L]{\nomUn, \nomDeux}
\fancyhead[C]{\devoir}
\fancyhead[R]{\sigle}

\newcommand{\nomUn}{C. Lungescu}
\newcommand{\nomcompletUn}{Christian Lungescu}
\newcommand{\matriculeUn}{\texttt{Matricule 20079725}}
\newcommand{\nomDeux}{F. Rouleau }%Félix}
\newcommand{\nomcompletDeux}{Félix Rouleau}%Félix}
\newcommand{\matriculeDeux}{\texttt{Matricule 20188552}}%Félix}
\newcommand{\devoir}{Travail pratique 2}
\newcommand{\sigle}{IFT 2035 E22}


\title{Travail Pratique 2}
\author{
\nomcompletUn\\
\matriculeUn\\
\nomcompletDeux\\
\matriculeDeux
}
\date{}

\begin{document}

\maketitle

\section*{Rapport}

Pour commencer, au début du projet, avant d'avoir même ajouté la première ligne 
de code, nous nous somme renseignés sur les plusieurs types d'allocation de mémoire
possibles. Après nous être renseignés, pour ce projet, nous avons choisi deux styles. Un style avec des \texttt{sizes} fixes
pour les structures, parce que leur \texttt{sizeof()} ne change pas. L'autre style était une avec 
un linked list qui pointe aux areas de mémoire ouvertes. Ces deux ensembles 
fonctionnent bien ensemble.\\
Il y avait l'option de réimplémenter \texttt{ptmalloc2} ou
quelque chose de ce style, mais c'était trop compliqué pour ce projet et c'est plus
efficace avec des styles avec des \texttt{ints}, \texttt{floats}, etc., et non juste une structure fixe
et des strings très variables en largeur.\\

Le langage \texttt{C} lui-même n'était pas très difficile à utiliser, mais il faut toujours faire
attention de ne pas implémenter de bugs, car c'est très facile de faire une erreur et se tirer dans 
le pied.\\

Le débogueur en \texttt{CLion} fonctionne extrêmement bien et nous
a beaucoup aidé à débugger le programme lorsqu'on en avait besoin. Ça nous a pris un
peu de temps pour comprendre qu'il fallait avoir \texttt{-d -O0} comme \texttt{CFLAGS} extra, mais après cela,
tout est bien allé.\\

Regarder les pointeurs en mémoire au aussi été très utile, nous ayant permis de
savoir ce que les bugs faisaient et de les régler. Une technique qui a également aidé fut
de garder un bout de papier avec un stylo près de notre ordinateur afin de dessiner et 
écrire nos problèmes sur papier.\\

Une chose qui nous a causé quelques problèmes initialement fut
de toujours caster au bon pointeur avant d'avancer par un certain nombre,
car il dépend du \texttt{sizeof()} du type.\\
Parfois, à cause de cela, nous multiplions accidentellement
et les offsets étaient mal calculés, ce qui causait des segmentation faults,
modifiait des données non supposées être modifiées en mémoire, et faisait également en sorte qu'il
n'y avait rien à la place où l'on se rendait. Après avoir réglé cela, il ne fut pas trop
difficile de se rendre compte des effets causés par cette erreur.\\

\pagebreak

\section*{Annexe}
Ce système utilise deux headers: un pour stocker les \texttt{String struct} et un pour
stocker les caractères des \texttt{String}. Les headers pointent vers des pages qui
augmentent de $2^index$ à chaque fois. Par exemple : \texttt{4096B}, \texttt{8192B}, \texttt{16384B}, \texttt{32768B}, etc.\\
Ceci fait en sorte que nous pouvons gérer beaucoup de mémoire dans un seul header.\\

À la base, le système utilise \texttt{sysconf(\_SC\_PAGESIZE)} bytes comme header de base.
Sur un système normal, cela correspond à 4096 bytes. 64 bit prend 8 bytes 
pour 1 pointeur, alors nous pouvons pointer à 512 pages différentes pour le 
\texttt{String struct} lui-même et pour le data. Alors jusqu'à $5 \times 10^157$ bytes de data
peuvent être stockés.\\

Chaque \texttt{(size\_t *)} du header pointe vers une page si elle est initialisée et \texttt{NULL}
sinon.\\

La première sorte de page dont nous allons discuter est la page de \texttt{String structs}.\\
Le premier \texttt{size\_t} dans cette page contient le nombre de \texttt{String structs} qui peuvent être
stockés. Ensuite, les prochains words, en fonction du nombre de \texttt{String structs} qui 
peuvent être stockés, contient des flags qui indiquent si la case est utilisée par un 
\texttt{String} ou non. L'allocation et la désallocation se font en flippant ce bit.\\
Les indexs sont calculés par le nombre de words utilisés pour des 

\begin{lstlisting}
flags * sizeof(size_t) * 8 + # du bit
\end{lstlisting}
Par la suite, nous pouvons aller à la fin du metadata 
en calculant où ça termine, et après, sauter \texttt{(String *)} index. Ici, nous pouvons facilement
faire nos \texttt{str->} pour accéder aux données. Dans celui-ci, on a \texttt{allocated size}, au cas où
ca différerait du \texttt{actual size}, on a \texttt{size}, le \texttt{size requested}, \texttt{char *data}, dans quelle page
de \texttt{String structs} qu'il est et finalement vers quelle page de \texttt{data} le \texttt{char *data} pointe.\\
Il y a très peu de overhead pour le metadata, mais ca vient avec un \texttt{cout} en 
efficacité de temps.\\

La deuxième sorte de page est la page de \texttt{data}. Celle-ci est très facile à comparer, car 
elle alloue des \texttt{size} variables. L'entrée est simplement un pointeur \texttt{(size\_t *)} qui
pointe vers la première place de libre (Linked list) dans la page. Ce qu'elle pointe
spécifiquement est un autre pointeur vers la prochaine place de la Linked list. Le \texttt{(size\_t)}
juste à côté du pointeur est lenombre  de bytes libres pour ce node. Malheureusement, 
l'amalgamage n'est pas implementé, donc il faut faire \texttt{str\_compact()} pour le même effet.\\

Nous ne stockons pas le nombre de bytes qu'est chaque page, car sa position dans le header
correspond à \texttt{(sysconf(\_SC\_PAGESIZE) * 2\^index)}.\\

Le stockage d'une chaîne de taille N est : \texttt{N Bytes (minimum 16) + sizeof(String)}.\\
Ceux-ci contiennent aussi 8 bytes pour le head pointer, 8 pour le nombre de strings dans 
la page \texttt{String}, et \texttt{ceil\_size\_t((double) max\_cells / (double) sizeof(size\_t) * 8)}.
Ceci se fait en plus que $O(1)$, mais moins que $O(n)$.\\

La libération prend un temps $O(1)$, toujours constant, car on ne fait que réécrire le metadata et flipper un bit.
La compactation prend un temps $O(n)$ avec le nombre de strings à compacter.\\

La fragmentation est un gros problème, seulement pour les \texttt{char*}, car il n'y a pas d'amalgamage. 
Cela fait en sorte qu'un bloc reste d'une certaine largeur après avoir été alloué.\\
Si nous avions assez de temps pour implémenter un système qui fonctionnait, ça ne serait plus
un gros problème.\\

\end{document}
