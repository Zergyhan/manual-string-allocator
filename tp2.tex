\documentclass{article}

\usepackage[utf8]{inputenc}
\usepackage{amsfonts}            %For \leadsto
\usepackage{amsmath}             %For \text
\usepackage{fancybox}            %For \ovalbox
\usepackage{hyperref}            %For \ovalbox
\usepackage[letterpaper,margin=1in,includehead]{geometry}

\title{Travail pratique \#2}
\author{IFT-2035}

\begin{document}

\maketitle

\newcommand \mML {\ensuremath\mu\textsl{ML}}
\newcommand \kw [1] {\textsf{#1}}
\newcommand \id [1] {\textsl{#1}}
\newcommand \punc [1] {\kw{`#1'}}
\newenvironment{outitemize}{
  \begin{itemize}
  \let \origitem \item \def \item {\origitem[]\hspace{-18pt}}
}{
  \end{itemize}
}

\section{Survol}

Ce TP a pour but de vous familiariser avec le langage C, les pointeurs,
la gestion de mémoire explicite.
%%
Comme pour le TP précédent, les étapes sont les suivantes:
\begin{enumerate}
\item Parfaire sa connaissance de C.
\item Lire et comprendre cette donnée.
\item Compléter le code fourni.
\item Écrire un rapport.  Il doit décrire \textbf{votre} expérience pendant
  les points précédents: problèmes rencontrés, surprises, choix que vous
  avez dû faire, options que vous avez sciemment rejetées, etc...  Le
  rapport ne doit pas excéder 5 pages plus 2 d'annexe (voir plus loin).
\end{enumerate}
%% 
Ce travail est à faire en groupes de 2 étudiants.  Le rapport (au format
\LaTeX) et le code sont à remettre par remise électronique avant la date
indiquée.  Aucun retard ne sera accepté.  Indiquez clairement votre nom au
début de chaque fichier.

Si un étudiant préfère travailler seul, libre à lui, mais l'évaluation de
son travail n'en tiendra pas compte.
Des groupes de 3 ou plus sont \textbf{exclus}.

\section{Une bibliothèque de gestion mémoire}

Vous devez écrire une bibliothèque de gestion mémoire pour des
chaînes de caractères qui fournisse les fonctions suivantes:
%%
\begin{verbatim}
   typedef struct String String;

   String *str_alloc   (size_t size);
   size_t  str_size    (String *str);
   char   *str_data    (String *str);
   void    str_free    (String *str);

   String *str_concat  (String *s1, String *s2);

   void    str_compact  (void);
   size_t  str_livesize (void);
   size_t  str_freesize (void);
   size_t  str_usedsize (void);
\end{verbatim}
%%
Libre à vous de définir \texttt{struct String} comme bon vous semble.
La fonction \id{str\_alloc} crée une nouvelle chaîne de taille \id{size}.
La fonction \id{str\_size} renvoie la taille d'une chaîne et la fonction
\id{str\_data} renvoie le tableau de bytes de la chaîne.
Finalement \id{str\_free} permet de libérer l'espace occupé par une chaîne
et \id{str\_compact} permet de compacter l'espace mémoire occupé par toutes
les chaînes de manière à réduire la fragmentation (c'est à dire réduire le
nombre de ``trous'' libres entre les espaces encore utilisés, pour les
remplacer par un seul (ou un petit nombre) de plus gros trou(s)).

Pour que la compaction soit possible, chaque \texttt{String} sera composé de
2 objets en mémoire: l'objet de type \texttt{struct String} qui a une taille
fixe indépendente de la longueur de la chaîne et qui ne peut pas être
déplacé par \id{str\_compact}, et un autre objet qui contient les bytes de
la chaîne de caractères (et cet objet-là peut être déplacé par
\id{str\_compact}).

La compaction présume que le reste du programme (le \emph{mutateur}),
n'utilise pas de \id{str\_data} pendant la compaction et de plus il doit
faire attention à ne plus utiliser les \id{str\_data} obtenus avant la
compaction vu qu'ils peuvent ne plus être valides.

Vous devez écrire le code dans un fichier \texttt{stralloc.c}.
Cette bibliothèque n'a pas le droit d'utiliser \id{malloc} pour obtenir de la
mémoire.  À la place, il faudra utiliser des fonctionnalités de plus bas
niveau, plus précisément \id{mmap} et \id{munmap}.  Ces fonctions peuvent
varier un peu entre différents systèmes d'exploitation, donc assurez-vous
que votre code fonctionne dans un système GNU/Linux raisonnablement récent.

\id{mmap} ne peut allouer que des multiples de pages (i.e. des multiples de
4KB par exemple).  Il faudra donc souvent les partager entre plusieurs
chaînes de caractères.  De plus \id{mmap} est relativement coûteux, donc il
est préférable de l'utiliser moins souvent en allouant plus de mémoire
à la fois.

Les fonctions \id{str\_livesize}, \id{str\_freesize}, et \id{str\_usedsize}
permettent de mesurer quel espace est utilisé pour quoi.

\section{Ce que vous devez faire}

\begin{itemize}
\item Écrire des tests dans \texttt{tests.c}.
\item Écrire le code dans \texttt{stralloc.c}.
\item Analyser le comportement de votre code (maximum 2 pages en annexe du
  rapport).  Il s'agit ici d'une analyse principalement théorique: mémoire
  effectivement consommée pour stocker une chaîne de taille N; nombre
  d'opérations nécessaires pour allouer une nouvelle chaîne, pour libérer
  une chaîne, pour compacter l'espace; impact de la fragmentation, ...
\item Écrire le reste du rapport.
\end{itemize}

Le code fourni dans \texttt{stralloc.h} ne peut pas être changé, à moins d'une
très bonne raison. \\
Remarquez que ce TP peut se décomposer en 3 parties:
\begin{enumerate}
\item Implanter seulement \id{str\_alloc}, \id{str\_size} et \id{str\_data};
  utiliser pour \id{str\_free} et \id{str\_compact} une implantation triviale
  qui ne fait rien.
\item Modifier le code pour implanter un vrai \id{str\_free}.
\item Modifier le code pour implanter un vrai \id{str\_compact}.
\end{enumerate}

\section{Remise}

Pour la remise, vous devez remettre trois fichiers (\texttt{stralloc.c},
\texttt{tests.c}, et \texttt{rapport.tex}) par la page Moodle (aussi
nommé StudiUM) du cours.  Assurez-vous que le rapport compile correctement
sur \texttt{ens.iro} (auquel vous pouvez vous connecter par SSH).

\section{Détails}
%%
\begin{itemize}
\item La note sera divisée comme suit: 25\% pour le rapport, 15\% pour les
  tests, 60\% pour le code C.

\item Un bon point de départ pour vous aider à décider comment gérer vos
  pages de mémoire est
  \\ à \url{http://www.memorymanagement.org/articles/alloc.html}.

\item Tout usage de matériel (code ou texte) emprunté à quelqu'un d'autre
  (ou trouvé sur le web) doit être dûment mentionné, sans quoi cela sera
  considéré comme du plagiat.

\item Chaque ligne de code doit faire moins de 80 caractères.
  Tout dépassement sera considéré comme une erreur.

\item Votre code doit compiler avec \texttt{gcc -Wall} sans
  générer plus d'avertissements que le code fourni.

\item Vérifiez la page web du cours, pour d'éventuels errata, et d'autres
  indications supplémentaires.

\item La note est basée d'une part sur des tests automatiques, d'autre part
  sur la lecture du code, ainsi que sur le rapport.  Le critère le plus
  important, est que votre code doit se comporter de manière correcte.
  Ensuite, vient la qualité du code: plus c'est simple, mieux c'est.
  S'il y a beaucoup de commentaires, c'est généralement un symptôme que le
  code n'est pas clair; mais bien sûr, sans commentaires le code (même
  simple) est souvent incompréhensible.
\end{itemize}

\end{document}

%% arch-tag: 25121aff-ecf7-418a-9b39-3f7100028f8b
